\documentclass[11pt, oneside, a4paper, titlepage]{article}

\usepackage[most]{tcolorbox} \usepackage{geometry} \geometry{ a4paper,
left=0.1cm, right=0.1cm, top=0.1cm, bottom=0.1cm }

\definecolor{titleBack}{RGB}{0,66,21}

\usepackage{hyperref}

\title{Adriano Ruiz Barbero}

\begin{document}

\tcbset{colframe=gray!95!black,colback=titleBack,arc=0mm}

\begin{tcolorbox} \begin{minipage}{4.5cm}
\hspace*{-0.3cm}
\includegraphics[width=4cm]{yo.png} 
\end{minipage}
\begin{minipage}{15cm} 
	\begin{center} 
		\Huge{\textcolor{white}{Adriano Ruiz Barbero}} \\ 
		\vspace*{0.5cm} 
		\Large{\textcolor{white}{\emph{Desarrollador de software}}} 
	\end{center} 
\end{minipage} 
\end{tcolorbox}

	\tcbset{colframe=white,colback=white,arc=0mm} \begin{tcolorbox}
		\begin{minipage}[t]{8cm} \vspace*{-0.5cm} \begin{tcolorbox}[grow to
			left by=0.6cm,colback=gray!25,colframe=white] \section*{Sobre mi}
			Soy un \textbf{desarrollador de software junior especializado en la
			programación gráfica y la gestión y administración de sistemas
			Linux.} Gracias a la educación alternativa de la que he sido
			beneficiario con 42 poseo una \textbf{gran capacidad de
			adaptabilidad y aprendizaje ademas de una
			gran capacidad para el trabajo en equipo.}

			42 es una famosa escuela de programación informática, creada y
			financiada por el multimillonario francés Xavier Niel (fundador de
			la empresa de telecomunicaciones Illiad) con varios socios, entre
			ellos Nicolas Sadirac (anterior director general de la escuela
			Epitech también conocida como European Institute of Technology).
			
			
			42 no tiene profesores, y está abierta 24/7. La formación se
			inspira en las nuevas formas modernas de enseñanza que incluyen la
			pedagogía entre compañeros y el aprendizaje basado en proyectos.

			\section*{Contacto} \begin{tabular}{r l} Tel: & +34 609 15 74 70 \\
			Email: & adriano@adrianoruiz.xyz \\ Web: & adrianoruiz.xyz \\
			\end{tabular}

		\section*{Habilidades} 
			\begin{itemize} 
				\item{c, python, shell, Ansible, unity, Jira, \LaTeX} 
				\item{programación gráfica} 
				\item{Gestión y configuración de Sistemas Linux} 
				\item{Gran capacidad para el aprendizaje y la enseñanza} 
			\end{itemize} 
		\section*{Lenguas} 
			\begin{itemize}
				\item{\textbf{Español:} Nativo}
				\item{\textbf{Inglés:} Nivel avanzado, más de 3 años en practica tanto en París como California}
			\end{itemize}
		\end{tcolorbox} \end{minipage} 
		\begin{minipage}[t]{11cm}
		\vspace*{-0.5cm} 
			\begin{tcolorbox}[grow to right
		by=0.75cm,colframe=white,colback=white] 
				\section{Experiencia Profesional} 

			\begin{itemize} 
				\item { \textbf{Practicas en GESERISK} \\
				\emph{Centro riesgo cibernético} \\ \emph{09/2020 al
				01/2021} \\ 
				\href{https://youtu.be/AlB1oqxZPXA}{Vídeo de
				presentación del resultado}} 
				\item { 
					\textbf{Trabajo a tiempo parcial en GESERISK} \\
					\emph{Centro riesgo cibernético} \\ 
					\emph{03/2021 al 04/2022} \\ 

					Mantenimiento de la web Domecq Academy y
					automatización de procesos en la empresa relacionados con
					ciberseguridad. 
				} 
				\item { 
					\textbf{Técnico IT y pedagógico en 42Málaga Fundación Telefónica} \\ 
					\emph{04/2022 al 09/2022} \\ 
					\textbf{DevOps y automatización, creación de
					herramientas para los estudiantes}

					Formar parte del proyecto 42 en Málaga en España. Durante
					este periodo formé parte del equipo IT y Pedago
					representante de \textbf{42 Málaga financiado por Fundación
					Telefónica, La Junta de Andalucía, La diputación de Málaga y el
					ayuntamiento de Malaga.}
					
					Gestión de picos de hasta 160 alumnos y los 300 imacs
					a disposición de estos en el espacio de trabajo.
				}
				\item {
					\textbf{Gestión en proyectos para el desarrollo
						social en el espacio La Noria de la Diputación de
						Málaga} \\
					\emph{Fuente de Innovación Social} \\
					\emph{04/2022 al 09/2022} \\
					Orientación y gestión de proyectos tecnológicos
					sociales del grupo de estudiantes de 42 beneficiarios de la
					beca de alojamiento de la Noria.
				}
			\end{itemize}
			
			\section{Educación y Certificados}
			\begin{itemize}
				\item
				{
					\textbf{42 Silicon Valley } \\
					\emph{noviembre de 2018 a  marzo de 2019} \\
				}
				\item
				{
					\textbf{42 París} \\
					\emph{marzo de 2019 a presente} \\
				}
				\item
				{
					\textbf{Certificado de Bachillerato y Selectividad} \\
					\emph{Junio 2018} \\
				}
			\end{itemize}
		\end{tcolorbox}
	\end{minipage}
\end{tcolorbox}

\newpage
\tcbset{freelance, colback=black!0!white, colframe=titleBack}
\begin{tcolorbox}[breakable,
				title={Adriano Ruiz Barbero \hfill Desarrollador de Software}]
				\section{Algunos de mis proyectos}
				\begin{itemize}
				\item
				{
					\textbf{Roger\_Skyline}
					\begin{enumerate}
						\item[--] Crear y configurar un servidor web en un sistema Linux.
						\item[--] Protección a ataques DOS y escaneo de puertos abiertos.
						\item[--] Creación de scripts para automatizar los servicios del servidor.
					\end{enumerate}
				}
				\item
				{
					\textbf{ft\_linux}
					\begin{enumerate}
						\item[--] Creación de mi propia distribución de Linux
					\end{enumerate}
				}
				\item
				{
					\textbf{boot2root} \\
					Proyecto de seguridad se nos da una maquina
					Linux y a traves de diversas tecnicas debemos
					ganar acceso a los usuarios hasta conseguir el
					root.
					\begin{enumerate}
						\item[--] Unix 
						\item[--] Security
						\item[--] Adaptación y creatividad
					\end{enumerate}
				}
				\item
				{
					\textbf{Maze\_Generator} \\
					\begin{enumerate}
						\item[--]Aplicación android para
							generar laberintos, 
						\item[--]Creado en Unity usando el algoritmo
							backtracking.

					\end{enumerate}
				}
				\end{itemize}
\end{tcolorbox}
\end{document}
